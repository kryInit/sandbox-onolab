Full waveform inversion (FWI)~\cite{FWI0,FWI1} aims to reconstruct subsurface properties from seismic data observed at multiple points.
These subsurface properties are used for geological research and resource exploration, including deposits of gas, oil, mineral, and groundwater~\cite{FWI1,FWIApplicationGroundwater0,FWIApplicationGroundwater1}.
In addition to geological fields, FWI has also been successfully applied to non-destructive testing in the medical and industrial fields~\cite{FWIApplicationNonDestructiveTesting0,FWIApplicationNonDestructiveTesting1}.

In FWI, the observation process of seismic data from subsurface properties is nonlinear and complex~\cite{FWI1}, making an analytic inverse transformation impossible.
An effective approach to address this is to formulate FWI as an optimization problem~\cite{FWI0,CustomFWI0,CustomFWI1,CustomFWI2,CustomFWI3,CustomFWI4,CustomFWI5}, such as minimizing the squared error between observed and modeled seismic data.
Since FWI is an ill-posed inverse problem, many methods have been proposed that incorporate Tikhonov~\cite{tikhonov} and Total Variation (TV)-type~\cite{TV,TGV} regularizations to capture the piecewise smoothness of subsurface properties~\cite{FWI-with-tikhonov-regularization,FWI-with-TV-regularization,FWI-with-directional-TV-regularization,FWI-with-high-order-TV-regularization,FWI-with-TGV-regularization}.
However, these regularizations require careful tuning of balance parameters between FWI objective value and these regularization values.
Instead of the regularizations, we focus on incorporating TV as a constraint into the FWI problem~\cite{FWI-with-TV-constraint,FWI-with-TV-constraint2,FWI-with-TV-constraint3,FWI-with-TV-constraint4}.
In contrast to the TV regularizations, the TV constraint has the advantage that its parameter can be determined independently of the objective function value like~\cite{constraint0,constraint1,constraint2,constraint3,constraint4}.
Specifically, the parameter of TV constraint can be determined based only on prior knowledge of the subsurface properties~\cite{constraints-vs-penalties-in-FWI}.
This makes the formulation and the reconstructed subsurface properties easier to interpret, which is beneficial for practical applications.

However, the TV constrained FWI problem is difficult to solve not only because of the nonlinearity of the observation process, but also because of the non-smoothness of the TV constraint.
To address this, conventional methods~\cite{FWI-with-TV-constraint,FWI-with-TV-constraint2,FWI-with-TV-constraint3,FWI-with-TV-constraint4} adjust the objective variable to satisfy the constraint at each step of an iterative optimization algorithm.
This requires an inner loop, which results in high computational cost.
In addition, the methods rely on approximations, such as linear or quadratic approximation and satisfying constraints outside the optimization method.
If the TV constrained FWI problem could be solved with neither inner loops nor approximations, more efficient and accurate reconstructions of subsurface properties would be possible.

In this paper, we propose a novel algorithm to solve the TV constrained FWI problem based on the primal-dual splitting (PDS) method.
Our algorithm addresses the challenges posed by both the nonlinearity of the observation process and the non-smoothness of the TV constraint without approximations, resulting in a more accurate reconstruction.
Furthermore, by handling the constraint without inner loops, our algorithm is significantly more efficient compared to existing methods.
We demonstrate that our algorithm efficiently handles the constraint while achieving accurate reconstruction.
