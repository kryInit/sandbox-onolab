% vec
\newcommand{\vecx}{\bm{x}}
\newcommand{\vecy}{\bm{y}}

% utils
\newcommand{\argmax}[1]{\underset{#1}{\mathrm{argmax}}}
\newcommand{\argmin}[1]{\underset{#1}{\mathrm{argmin}}}
\newcommand{\minimize}[1]{\underset{#1}{\mathrm{min}}}
\newcommand{\maximize}[1]{\underset{#1}{\mathrm{min}}}
\newcommand{\Norm}[2]{\lVert #1 \rVert}
\newcommand{\LOneNorm}[1]{\lVert #1 \rVert _1}
\newcommand{\LTwoNorm}[1]{\lVert #1 \rVert _2}
\newcommand{\LOneTwoNorm}[1]{\lVert #1 \rVert _{1,2}}
\newcommand{\TV}[1]{\mathrm{TV}(#1)}
\newcommand{\realNumber}{\mathbb{R}}
\newcommand{\intN}{\mathrm{N}}
\newcommand{\intM}{\mathrm{M}}

\newcommand{\LOneTwoNormDefinition}{\LOneTwoNorm{\vecx} \coloneq  \sum_{\mathfrak{g} \in \mathfrak{G}} \LTwoNorm{\vecx_\mathfrak{g}}}
\newcommand{\TotalVariationDefinition}{\TV{\vecx} \coloneq \LOneTwoNorm{\diffOperator \vecx} = \sum_{i=1}^{\intN} \sqrt{d_{h,i}^2 + d_{v,i}^2}}
\newcommand{\conjugateFunctionDefinition}{f^*(\vecx) \coloneq \sup_{\vecy \in \realNumber^N} \left\{ \vecy^T \vecx - f(\vecy) \right\}}

% indicator function
\newcommand{\indicatorFunction}[2]{\iota_{#1}(#2)}
\newcommand{\indicatorFunctionDefinition}{\indicatorFunction{C}{\vecx} \coloneq \begin{cases} 0 & \text{if } \vecx \in C, \\ \infty & \text{otherwise}. \end{cases} }

% proximity operator
\newcommand{\proximityOperator}[2]{\mathrm{prox}_{#1}(#2)}
\newcommand{\proximityOperatorDefinition}{\proximityOperator{ \gamma f }{\vecx} := \argmin{\vecy \in \realNumber^N} \left\{ f(\vecy) + \frac{1}{2 \gamma} \LTwoNorm{\vecy - \vecx}^2 \right\}}
\newcommand{\proximityOperatorDefinitionWithConvexConjugateFunction}{\proximityOperator{ \gamma f^* }{\vecx} = \vecx - \gamma \proximityOperator{ f / \gamma }{\vecx / \gamma} }
\newcommand{\projBox}[1]{ P_{[a,b]^N}(#1) }
\newcommand{\projBoxSolution}{ \projBox{\vecx} = \text{min}( \text{max} (\vecx, a), b) }
\newcommand{\projLOneTwoBall}[1]{P_{ \{ \bm a \mid \LOneTwoNorm{ \bm a } \le \alpha \} }(#1)}
\newcommand{\projLOneTwoBallSolution}{ \projLOneTwoBall{\vecx} = [ \bm p_1^T, \ldots, \bm p_N^T ]^T }
\newcommand{\projLOneTwoBallSolutionWhere}{
    \begin{aligned}
        & \bm p_i = \begin{cases}
            0                                              & \text{if} \ \LTwoNorm{\vecx_i} = 0, \\
            \bm \beta_i \frac {\vecx_i} {\LTwoNorm{\vecx}} & \text{otherwise},
        \end{cases} \\
        & \bm \beta = P_{ \{ \bm a \mid \LOneNorm{ \bm a } \le \alpha \} }({[ \LTwoNorm{ \vecx_1^T }, \ldots, \LTwoNorm{ \vecx_N^T } ]}). \\
    \end{aligned}
}
\newcommand{\projLOneBall}[1]{P_{ \{ \bm a \mid \LOneNorm{ \bm a } \le \alpha \} }(#1)}
\newcommand{\projLOneBallSolution}{\projLOneBall{\vecx} = \text{SoftThrethold}(\vecx, \beta)}
\newcommand{\projLOneBallSolutionWhere}{
    \begin{aligned}
        & \bm x_{\text{abs}} = \text{abs}(\vecx), \\
        & \bm y              = \text{sort}_{\text{desc}}(\vecx_{\text{abs}}), \\
        & \beta'             = \text{max} \{ \frac 1 i ((\sum_{j=1}^i \bm y_j) - \alpha) \mid i = 1, \ldots, N \}, \\
        & \beta              = \text{max} \{ \beta', 0 \}. \\
    \end{aligned}
}


% Primal-Dual Splitting
\newcommand{\PDSPrimal}{\minimize { \vecx \in \realNumber^N } \left\{ f(\vecx) + g(\vecx) + h(\bm{L} \vecx) \right\} }
\newcommand{\PDSDual}{\minimize { \vecy \in \realNumber^M } \left\{ (f+g)^*(-\bm{L}^T \vecy) + h^*(\vecy) \right\} }
\newcommand{\PDSSubStep}{
    \begin{aligned}
        & \vecx^{(k+1)} = \proximityOperator{\gamma_1 g}{\vecx^{(k)} - \gamma_1( \nabla f(\vecx^{(k)}) + \bm{L}^T \vecy^{(k)} )}, \\
        & \vecy^{(k+1)} = \proximityOperator{\gamma_2 h^*}{\vecy^{(k)} + \gamma_2 \bm{L} (2\vecx^{(k+1)} - \vecx^{(k)}) }, \\
    \end{aligned}
}

% seismic + related proposed method
\newcommand{\diffOperator}{\mathbf{D}}
\newcommand{\velModel}{\bm{m}}
\newcommand{\seismicData}{\bm{u}}

% FWI objective
\newcommand{\FWIObjectiveDefinition}{ E(\velModel) = \frac {1} {2} \LTwoNorm { \seismicData_{\mathrm{obs}} - \seismicData_{\mathrm{cal}(\velModel)} }^2 }
\newcommand{\FWIGradientDefinition}{ \nabla E(\velModel) = \seismicData_{\mathrm{obs}} - \nabla \seismicData_{\mathrm{cal}(\velModel)} }
\newcommand{\FWIObjectiveWithTVConstraint}{ E(\velModel) \ \ \ \text{s.t.} \ \ \LOneTwoNorm{\diffOperator \velModel} \le \alpha \ , \ \velModel \in [a,b]^N }
\newcommand{\FWIObjectiveWithTVConstraintWithIndicatorFunction}{ E(\velModel) + \indicatorFunction{\LOneTwoNorm{\cdot} \le \alpha}{\diffOperator \velModel} + \indicatorFunction{[a,b]^N}{\velModel} }
\newcommand{\FWIWithPDS}{
\left \lbrack \ \
    \begin{aligned}
        & \widetilde{\velModel}^{(k+1)} = \velModel^{(k)} - \gamma_1( \nabla E(\velModel^{(k)}) + \bm{D}^T \vecy^{(k)} ) \\
        & \velModel^{(k+1)}             = \projBox{\widetilde{\velModel}^{(k+1)}} \\
        & \widetilde{\vecy}^{(k+1)}     = \vecy^{(k)} + \gamma_2 \bm{D} (2\velModel^{(k+1)} - \velModel^{(k)}) \\
        & \vecy^{(k+1)}                 = \widetilde{\vecy}^{(k+1)} - \gamma_2 \projLOneTwoBall{\frac 1 {\gamma_2} {\widetilde{\bm y}^{(k+1)}}}
    \end{aligned}
\right.
}
\newcommand{\FWIWithGradient}{ \velModel^{(k+1)} = \velModel^{(k)} - \gamma( \nabla E(\velModel^{(k)}) ) }



