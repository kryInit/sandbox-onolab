制約付きFWIの目的関数を、PDSを用いて求解します
TV制約, box制約付きFWIは以下の式で表されます
\begin{equation} \label{eq:FWIObjectiveWithTVConstraint}
\FWIObjectiveWithTVConstraint
\end{equation}

PDSを適用するために、制約をindicator functionとして目的関数に組み込みます
\begin{equation} \label{eq:FWIObjectiveWithTVConstraintWithIndicatorFunction}
\FWIObjectiveWithTVConstraintWithIndicatorFunction
\end{equation}

PDSを適用し、得られたstepは以下の通りとなります
%\begin{equation} \label{eq:PDSSubStepX}
%    \PDSSubStepX
%    \nota
%\end{equation}
