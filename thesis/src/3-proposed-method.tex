As shown in Fig.\ref{fig:salt-model}, the velocity model of the Salt is piecewise smooth.
Therefore, we introduce the TV constraint to achieve more accurate reconstruction.
Also, by introducing the box constraint, we can ensure that the velocity model does not take invalid values, and we show the flexibility of incorporating constraints using PDS.

We minimize the objective function of the TV and box constrained FWI, which is expressed as follows:
\begin{equation} \label{eq:FWIObjectiveWithTVConstraint} \argmin{\velModel \in \realNumber^N} \ \ \FWIObjectiveWithTVConstraint \end{equation}
where $\alpha \in \realNumber$ is the upper bound of the $l_{1,2}$ norm, and $a, b \in \realNumber$ are the lower and upper bounds of the velocity model value, respectively.

The constraints can be incorporated into the objective function as indicator functions:
\begin{equation} \label{eq:FWIObjectiveWithTVConstraintWithIndicatorFunction} \argmin{\velModel \in \realNumber^N} \ \ \FWIObjectiveWithTVConstraintWithIndicatorFunction \end{equation}

The proximity operator of $\iota_{\LOneTwoNorm{\cdot} \le \alpha}$ and $\iota_{[a,b]^N}$ can be computed efficiently.
Therefore, these functions of $E$, $\iota_{[a,b]^N}$ and $\iota_{\LOneTwoNorm{\cdot} \le \alpha}$ correspond to $f$, $g$ and $h$ in \eqref{eq:PDSPrimalEq}, respectively, $\diffOperator$ is corresponds to $\bm{L}$, and the problem~\eqref{eq:FWIObjectiveWithTVConstraintWithIndicatorFunction} can be solved using PDS.
The iterative procedures are as follows:
\begin{equation} \label{eq:FWIWithPDS} \FWIWithPDS \notag \end{equation}


The following are the proximity operators of indicator function of the box constraint and the $l_{1,2}$ norm upper bound constraint.
\begin{equation} \label{eq:ProximityOperatorForBoxConstraint} \projBoxSolution. \end{equation}

%The proximity operator for the $l_{1,2}$ norm upper bound constraint is expressed as follows~\cite{L12-ball-projection}:
\begin{equation} \label{eq:ProximityOperatorForL12Ball} \projLOneTwoBallSolution, \end{equation}
where
\begin{equation} \label{eq:ProximityOperatorForL12BallWhere} \projLOneTwoBallSolutionWhere \notag \end{equation}

The proximity operator for the $l_1$ norm upper bound constraint is expressed as follows~\cite{L1-ball-projection}(faster algorithms are also proposed~\cite{fast-L1-ball-projection}):
\begin{equation} \label{eq:ProximityOperatorForL1Ball}  \projLOneBallSolution, \end{equation}
where
\begin{equation} \label{eq:ProximityOperatorForL1BallWhere} \projLOneBallSolutionWhere \notag \end{equation}

