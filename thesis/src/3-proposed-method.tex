制約付きFWIの目的関数を、PDSを用いて求解します
TV制約, box制約付きFWIは以下の式で表されます
\begin{equation} \label{eq:FWIObjectiveWithTVConstraint} \FWIObjectiveWithTVConstraint \end{equation}
PDSを適用するために、制約をindicator functionとして目的関数に組み込みます
\begin{equation} \label{eq:FWIObjectiveWithTVConstraintWithIndicatorFunction} \FWIObjectiveWithTVConstraintWithIndicatorFunction \end{equation}
PDSを適用し、得られたstepは以下の通りとなります
\begin{equation} \label{eq:FWIWithPDS} \FWIWithPDS \notag \end{equation}
ここで、$\projBox{\cdot}$ , $\projLOneTwoBall{\cdot}$ はそれぞれ\eqref{eq:ProximityOperatorForBoxConstraint},\eqref{eq:ProximityOperatorForL12Ball}により,
勾配${\nabla E(\cdot)}$はadjoint-state method\cite{FWI-gradient}により計算可能です.
