We solve the objective function of the TV and box constrained FWI using PDS, which is expressed by the following equation:
\begin{equation} \label{eq:FWIObjectiveWithTVConstraint} \FWIObjectiveWithTVConstraint \end{equation}
where $\alpha$ is the upper bound of the $l_{1,2}$ norm, and $a$ and $b$ are the lower and upper bounds of the velocity model, respectively.

To apply PDS, the constraints are integrated into the objective function as indicator functions:
\begin{equation} \label{eq:FWIObjectiveWithTVConstraintWithIndicatorFunction} \FWIObjectiveWithTVConstraintWithIndicatorFunction \end{equation}

The steps obtained by applying PDS are as follows:
\begin{equation} \label{eq:FWIWithPDS} \FWIWithPDS \notag \end{equation}
Here, $\projBox{\cdot}$ and $\projLOneTwoBall{\cdot}$ are given by\eqref{eq:ProximityOperatorForBoxConstraint} and \eqref{eq:ProximityOperatorForL12Ball}, respectively, and the gradient ${\nabla E(\cdot)}$ can be computed using the adjoint-state method ~\cite{FWI-gradient}.

