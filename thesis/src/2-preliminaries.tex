\subsection{Mathematical Tools}\label{subsec:mathematical-tools}

Throughout this paper, we denote vector and matrix by boldface lowercase letter (e.g., $\vecx$) and boldface uppercase letter (e.g., $\bm{X}$), respectively.
The operator $l_{X}$ norm of a vector and matrix is denoted by $\Norm{\cdot}__{X}$.

For $\vecx \in \realNumber^{\intN}$, the mixed $l_{1,2}$ norm is defined as follows:
\begin{equation} \label{eq:L12NormDefinitionEq} \LOneTwoNormDefinition \end{equation}
where $\mathfrak{G}$ is a set of disjoint index sets, and $\vecx_{\mathfrak{g}}$ is the subvector of $\vecx$ indexed by $\mathfrak{g}$.

For $\vecx \in \realNumber^{\intN}$, the total variation (TV)~\cite{TV} is defined as follows:
\begin{equation} \label{eq:TVDefinitionEq} \TotalVariationDefinition \end{equation}
where $d_{h,i}$ and $d_{v,i}$ are the horizontal and vertical differences of the $i$-th element of $\vecx$, respectively, when vector $\vecx$ is considered as a matrix.

For proper lower-semicontinuous convex function $f \in \realNumber^N \to \realNumber$ and $\vecx \in \realNumber^N$, the convex conjugate function is defined as follows:
\begin{equation} \label{eq:ConjugateFunctionDefinitionEq} \conjugateFunctionDefinition \end{equation}

For a set $C \subset \realNumber^N$ and $\vecx \in \realNumber^N$, the indicator function is defined as follows:
\begin{equation} \label{eq:IndicatorFunctionDefinitionEq} \indicatorFunctionDefinition \end{equation}

For $\gamma > 0$, $f \in \realNumber^N \to \realNumber$ and $\vecx \in \realNumber^N$, the proximity operator is defined as follows:
\begin{equation} \label{eq:ProximityOperatorDefinitionEq} \proximityOperatorDefinition \end{equation}

Define the proximity operator for the indicator function as $P_C$ as follows.
\begin{equation} \label{eq:ProximityOperatorDefinitionWithIndicatorFunctionEq}
\proximityOperator{ \gamma \indicatorFunction{C}{\cdot} }{\vecx} = P_C(\vecx) \coloneq \argmin{\vecy \in C} \LTwoNorm{\vecy - \vecx}
\end{equation}

The proximity operator for the specific function used in this paper is given below.
%The proximity operator for the convex conjugate function is expressed as follows~\cite[Theorem 3.1 (ii)]{prox-convex-conjugate-function}:
\begin{equation} \label{eq:ProximityOperatorDefinitionWithConvexConjugateFunctionEq} \proximityOperatorDefinitionWithConvexConjugateFunction \end{equation}

%The proximity operator for the box constraint is expressed as follows.
\begin{equation} \label{eq:ProximityOperatorForBoxConstraint} \projBoxSolution \end{equation}

%The proximity operator for the $l_1$ norm upper bound constraint is expressed as follows~\cite{L1-ball-projection}(faster algorithms are also proposed~\cite{fast-L1-ball-projection}):
\begin{equation} \label{eq:ProximityOperatorForL1Ball}  \projLOneBallSolution \end{equation}
where
\begin{equation} \label{eq:ProximityOperatorForL1BallWhere} \projLOneBallSolutionWhere \notag \end{equation}

%The proximity operator for the $l_{1,2}$ norm upper bound constraint is expressed as follows~\cite{L12-ball-projection}:
\begin{equation} \label{eq:ProximityOperatorForL12Ball} \projLOneTwoBallSolution \end{equation}
where
\begin{equation} \label{eq:ProximityOperatorForL12BallWhere} \projLOneTwoBallSolutionWhere \notag \end{equation}

The proof of equation \eqref{eq:ProximityOperatorDefinitionWithConvexConjugateFunctionEq}, \eqref{eq:ProximityOperatorForL1Ball} and \eqref{eq:ProximityOperatorForL12Ball} can be found in~\cite[Theorem 3.1 (ii)]{prox-convex-conjugate-function},~\cite{L1-ball-projection},~\cite{L12-ball-projection} accordingly.
There is a faster algorithm than \eqref{eq:ProximityOperatorForL1Ball}~\cite{fast-L1-ball-projection}.

\subsection{Primal-Dual Splitting Algorithm}\label{subsec:primal-dual-splitting-algorithm}
The Primal-Dual Splitting (PDS) Algorithm~\cite{PDS0,PDS1,PDS2,PDS3} is applied to the following problem:
\begin{equation} \label{eq:PDSPrimalEq} \PDSPrimal \end{equation}
where $\bm{L} \in \realNumber^{\intM \times \intN}$, $f$ is differentiable convex function and $g,h$ are convex functions whose proximity operator can be computed efficiently.

The PDS algorithm solve by iteratively updating the following:
\begin{equation} \label{eq:PDSSubStepX} \PDSSubStepX \end{equation}
\begin{equation} \label{eq:PDSSubStepY} \PDSSubStepY \end{equation}
where $\gamma_1, \gamma_2 \in \realNumber$ are step sizes.
%for more details, please refer to~\cite{PDS2}.

\subsection{Full Waveform Inversion}\label{subsec:full-waveform-inversion}
An objective function of FWI is defined as follows\cite{FWI0}:
\begin{equation} \label{eq:FWIObjective} \FWIObjectiveDefinition \end{equation}
where $\velModel \in \realNumber^{N}$ is velocity model representing subsurface properties, $\seismicData_{\mathrm{obs}} \in \realNumber^{M}$ is the observed seismic data, and $\seismicData_{\mathrm{cal}(\velModel)}$ is the calculated seismic data with the velocity model.
$N$ is the number of grid points, and $M$ is the number of observed signals.
In general, velocity model is 2D or 3D grid data, but for simplicity we consider flattened 1D vector.
