\subsection{Mathematical Tools} \label{subsec:mathematical-tools}

Throughout this paper, we denote vectors and matrices by bold lowercase letters (e.g., $\vecx$) and bold uppercase letters (e.g., $\bm{X}$), respectively.

For $\vecx \in \realNumber^{\intN}$, the mixed $l_{1,2}$ norm is defined as follows:
\begin{equation} \label{eq:L12NormDefinitionEq} \LOneTwoNormDefinition, \end{equation}
where $\mathfrak{G}$ is a set of disjoint index sets, and $\vecx_{\mathfrak{g}}$ is the subvector of $\vecx$ indexed by $\mathfrak{g}$.

For $\vecx \in \realNumber^{\intN}$, the total variation (TV)~\cite{TV} is defined as follows:
\begin{equation} \label{eq:TVDefinitionEq} \TotalVariationDefinition, \end{equation}
where $d_{h,i}$ and $d_{v,i}$ are the horizontal and vertical differences of the $i$-th element of $\vecx$, respectively, when the vector $\vecx$ is considered as a matrix.



\subsection{Proximal Tools} \label{subsec:proximal-tools}

We denote by $\Gamma_0(\realNumber^N)$ the set of proper lower-semicontinuous convex functions $\realNumber^N \to (- \infty, \infty]$.

For $\gamma > 0$ and $f \in \Gamma_0(\realNumber^N)$, the proximity operator is defined as follows:
\begin{equation} \label{eq:ProximityOperatorDefinitionEq} \proximityOperatorDefinition. \end{equation}

For $f \in \Gamma_0(\realNumber^N)$, the convex conjugate function $f^*$ is defined as follows:
\begin{equation} \label{eq:ConjugateFunctionDefinitionEq} \conjugateFunctionDefinition. \end{equation}

The proximity operator for the convex conjugate function is expressed as follows~\cite[Theorem 3.1 (ii)]{prox-convex-conjugate-function}:
\begin{equation} \label{eq:ProximityOperatorDefinitionWithConvexConjugateFunctionEq} \proximityOperatorDefinitionWithConvexConjugateFunction. \end{equation}

For a nonempty closed convex set $C \subset \realNumber^N$, the indicator function $\iota_C: \realNumber^N \to (- \infty, \infty] $ is defined as follows:
\begin{equation} \label{eq:IndicatorFunctionDefinitionEq} \indicatorFunctionDefinition \end{equation}

The proximity operator of $\iota_C$ is the projection onto $C$, given by
%Define the proximity operator for the indicator function as $P_C$ as follows.
\begin{equation} \label{eq:ProximityOperatorDefinitionWithIndicatorFunctionEq}
\proximityOperator{ \gamma \iota_{C} }{\vecx} = P_C(\vecx) \coloneq \argmin{\vecy \in C} \ \LTwoNorm{\vecy - \vecx}.
\end{equation}



\subsection{Primal-Dual Splitting Algorithm} \label{subsec:primal-dual-splitting-algorithm}

The Primal-Dual Splitting algorithm (PDS)~\cite{PDS0,PDS1,PDS2,PDS3} is applied to the following problem:
\begin{equation} \label{eq:PDSPrimalEq} \PDSPrimal, \end{equation}
where $\bm{L} \in \realNumber^{\intM \times \intN}$, $f$ is a differentiable convex function and $g,h$ are convex functions whose proximity operator can be computed efficiently.

PDS solves Prob.~\eqref{eq:PDSPrimalEq} by iteratively updating the following:
\begin{equation} \label{eq:PDSSubStep} \PDSSubStep \end{equation}
where $\gamma_1, \gamma_2 \in \realNumber$ are step sizes.


\subsection{Full Waveform Inversion} \label{subsec:full-waveform-inversion}

Typically, FWI is treated as an optimization problem as follows\cite{FWI0}:
\begin{equation} \label{eq:FWIObjective} \argmin{\velModel \in \realNumber^N} \ \ \FWIObjectiveDefinition, \end{equation}
where $\velModel \in \realNumber^{N}$ is the velocity model representing subsurface properties, $\seismicData_{\mathrm{obs}} \in \realNumber^{M}$ is the observed seismic data, $\seismicData_{\mathrm{cal}} : \realNumber^{N} \rightarrow \realNumber^{M}$ is the observation process, and $\seismicData_{\mathrm{cal}}(\velModel)$ is the modeled seismic data with the velocity model.
$N$ is the number of grid points, and $M$ is the total data size of the observed seismic data, defined as the total product of the number of waveform sources, time samples, and receivers.

The observation process $\seismicData_{\mathrm{cal}}$ is nonlinear and complex, making it difficult to analytically derive the optimal solution.
However, the gradient $\nabla E$ can be computed numerically by simulating the wave equation using the adjoint-state method~\cite{FWI-gradient}.
