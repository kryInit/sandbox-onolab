%\begin{figure}[t]
    \centering
    \begin{tabular}{m{76mm} m{10mm}} % 2列レイアウト
        \includegraphics[width=80mm]{public/full_true_vm} &
        \raisebox{1.5mm}{\includegraphics[height=23mm]{public/color-bar}}
    \end{tabular}
    \vspace{-2mm}
    \caption{The velocity model of the Salt [km/s]}
    \vspace{-2mm}
    \label{fig:salt-model}
\end{figure}


We introduce the TV and box constraint into the FWI problem to achieve more accurate reconstruction.
As shown in Fig.~\ref{fig:experiment-data}, the velocity model is piecewise smooth, thus introducing the TV constraint to achieve a more accurate reconstruction.
The box constraint ensures that the velocity model remains within valid ranges.

The optimization problem of the TV and box constrained FWI is formulated as follows:
\begin{equation} \label{eq:FWIObjectiveWithTVConstraint} \argmin{\velModel \in \realNumber^N} \ \ \FWIObjectiveWithTVConstraint \end{equation}
where $\alpha \ge 0$ is the upper bound of the $\ell_{1,2}$ norm, and $u \ge l > 0$ are the upper and lower bounds of the velocity model values, respectively.
By incorporating TV as a constraint, the parameter $\alpha$ can be determined independently of other terms or constraints, which has been highlighted as an advantage in prior works~\cite{constraint0,constraint1,constraint2,constraint3,constraint4,constraints-vs-penalties-in-FWI}.
This separation makes it possible to directly control smoothness according to $\alpha$, providing a clearer interpretation of the reconstructed subsurface properties.

The constraints can be incorporated into the objective function as indicator functions:
\begin{equation} \label{eq:FWIObjectiveWithTVConstraintWithIndicatorFunction} \argmin{\velModel \in \realNumber^N} \ \ \FWIObjectiveWithTVConstraintWithIndicatorFunction, \end{equation}
where
\begin{equation}
    \label{eq:LOneTwoBallDefinition}
    \begin{aligned}[b]
        \BoxBall & \coloneqq [l,u]^N, \\
        \LOneTwoBall & \coloneqq \LOneTwoBallSetDefinition.
    \end{aligned}
\end{equation}


The proximity operator of $\iota_{\BoxBall}$ and $\iota_{\LOneTwoBall}$ can be computed efficiently.
Therefore, these functions of $E$, $\iota_{\BoxBall}$ and $\iota_{\LOneTwoBall}$ correspond to $f$, $g$ and $h$ in \eqref{eq:PDSPrimalEq}, respectively, $\diffOperator$ is corresponds to $\bm{L}$, so the problem~\eqref{eq:FWIObjectiveWithTVConstraintWithIndicatorFunction} can be solved using PDS.
We show the detailed algorithm in Algorithm~\ref{alg:FWIWithPDS}.
%\begin{equation} \label{eq:FWIWithPDS} \FWIWithPDS \notag \end{equation}
\begin{algorithm}[t]
    \caption{PDS based solver for~\eqref{eq:FWIObjectiveWithTVConstraintWithIndicatorFunction}}\label{alg:FWIWithPDS}
    \begin{algorithmic}[1]
        \Statex \textbf{Input:} $ \velModel^{(0)}, \vecy^{(0)}, \gamma_0 > 0, \gamma_1 > 0 $
        \While {A stopping criterion is not satisfied}
            \State $\widetilde{\velModel} \leftarrow \FWIWithPDSStepMTmp $
            \State $\velModel^{(k+1)}     \leftarrow \FWIWithPDSStepM $
            \State $\widetilde{\vecy}     \leftarrow \FWIWithPDSStepYTmp $
            \State $\vecy^{(k+1)}         \leftarrow \FWIWithPDSStepY $
        \EndWhile
        \Statex \textbf{Output:} $\velModel^{(k)}$
    \end{algorithmic}
\end{algorithm}


The proximity operators of $\iota_{\BoxBall}$, $\iota_{\LOneTwoBall}$, that is, the projection onto $\BoxBall$ and ${\LOneTwoBall}$ are calculated by
\begin{equation} \label{eq:ProximityOperatorForBoxConstraint} \projBoxSolution, \end{equation}
\begin{equation} \label{eq:ProximityOperatorForL12Ball} \projLOneTwoBallSolution \end{equation}
where
\begin{equation} \label{eq:ProximityOperatorForL12BallWhere} \projLOneTwoBallSolutionWhere, \end{equation}
and $\mathfrak{g}_i$ is an index set corresponding to the horizontal and vertical differences of the $i$-th element of $\velModel$.

The proximity operator for the $\ell_1$ norm upper bound constraint is provided in~\cite{L1-ball-projection}:
%The proximity operator for the $l_1$ norm upper bound constraint is expressed as follows~\cite{L1-ball-projection}:
%\begin{equation} \label{eq:ProximityOperatorForL1Ball}  \projLOneBallSolution, \end{equation}
%where
%\begin{equation}
%    \label{eq:ProximityOperatorForL1BallWhere}
%    \begin{aligned}[b]
%        & \bm x_{\text{abs}} = \text{abs}(\vecx), \\
%        & \bm y              = \text{sort}_{\text{desc}}(\vecx_{\text{abs}}), \\
%        & \beta'             = \text{max} \left\{ \frac 1 i \left(\left(\sum_{j=1}^i \bm y_j\right) - \alpha\right) \middle| \ i = 1, \ldots, N \right\}, \\
%        & \beta              = \text{max} \left\{ \beta', 0 \right\}. \\
%    \end{aligned}
%\end{equation}

%\subsection{Computational Cost of Our Algorithm} \label{subsec:computational-cost-of-out-algorithm}
\vspace{4mm}
\noindent \textit{Remark}: Computational Cost of Our Algorithm
\vspace{2mm}

%Our algorithm efficiently incorporates constraints without relying on approximations or inner loops, enabling fast execution.
%The most computationally dominant part of the algorithm is the computation of the gradient $\nabla E$, which involves simulating the wave equation along the time axis.
%Specifically, its computational complexity is $O(S\, TN)$, where $S$ is the number of waveform sources, $T$ is the number of time samples, and $N$ is the number of grid points in the velocity model.
%In contrast, the computational cost of enforcing the constraints is significantly lower, because its computational complexity is $O(N \log N)$.
%This ensures that the overhead introduced by the constraints is negligible compared to the dominant $\nabla E$ computation, making the overall algorithm highly efficient.

In our algorithm, the additional cost to enforce the TV and box constraints at each iteration is dominated by the computation of $\LOneTwoBall$, which has a computational complexity of $O(N \log N)$, where $N$ is the number of grid points in the velocity model.
In contrast, existing methods rely on inner loops to enforce the same constraints, with a computational complexity of $O(XY)$, where $X$ is the number of iterations in the inner loop and $Y$ is the computational cost per iteration.
Because $Y$ scales larger than $N \log N$, our approach achieves significantly faster performance than existing methods.
