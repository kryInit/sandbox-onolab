%
%  信号処理シンポジウム用テンプレート
%

\documentclass[10pt]{jarticle}
\usepackage{sips}
\usepackage{bm}
\usepackage{latexsym}
\usepackage{amsmath}
\usepackage{amsfonts}
\usepackage{mathtools}
%\usepackage[dvips]{graphicx}
\usepackage[dvipdfmx]{graphicx}
\usepackage{setspace}
\usepackage{caption}
\usepackage{array}
\usepackage{multirow}
\usepackage{multicol}


% vec
\newcommand{\vecx}{\bm{x}}
\newcommand{\vecy}{\bm{y}}

% utils
\newcommand{\argmax}[1]{\underset{#1}{\mathrm{argmax}}}
\newcommand{\argmin}[1]{\underset{#1}{\mathrm{argmin}}}
\newcommand{\minimize}[1]{\underset{#1}{\mathrm{min}}}
\newcommand{\maximize}[1]{\underset{#1}{\mathrm{min}}}
\newcommand{\Norm}[2]{\lVert #1 \rVert}
\newcommand{\LOneNorm}[1]{\lVert #1 \rVert _1}
\newcommand{\LTwoNorm}[1]{\lVert #1 \rVert _2}
\newcommand{\LOneTwoNorm}[1]{\lVert #1 \rVert _{1,2}}
\newcommand{\TV}[1]{\mathrm{TV}(#1)}
\newcommand{\realNumber}{\mathbb{R}}
\newcommand{\intN}{\mathrm{N}}
\newcommand{\intM}{\mathrm{M}}

\newcommand{\LOneTwoNormDefinition}{\LOneTwoNorm{\vecx} \coloneq  \sum_{\mathfrak{g} \in \mathfrak{G}} \LTwoNorm{\vecx_\mathfrak{g}}}
\newcommand{\TotalVariationDefinition}{\TV{\vecx} \coloneq \LOneTwoNorm{\diffOperator \vecx} = \sum_{i=1}^{\intN} \sqrt{d_{h,i}^2 + d_{v,i}^2}}
\newcommand{\conjugateFunctionDefinition}{f^*(\vecx) \coloneq \sup_{\vecy \in \realNumber^N} \left\{ \vecy^T \vecx - f(\vecy) \right\}}

% indicator function
\newcommand{\indicatorFunction}[2]{\iota_{#1}(#2)}
\newcommand{\indicatorFunctionDefinition}{\indicatorFunction{C}{\vecx} \coloneq \begin{cases} 0 & \text{if } \vecx \in C, \\ \infty & \text{otherwise}. \end{cases} }

% proximity operator
\newcommand{\proximityOperator}[2]{\mathrm{prox}_{#1}(#2)}
\newcommand{\proximityOperatorDefinition}{\proximityOperator{ \gamma f }{\vecx} := \argmin{\vecy \in \realNumber^N} \left\{ f(\vecy) + \frac{1}{2 \gamma} \LTwoNorm{\vecy - \vecx}^2 \right\}}
\newcommand{\proximityOperatorDefinitionWithConvexConjugateFunction}{\proximityOperator{ \gamma f^* }{\vecx} = \vecx - \gamma \proximityOperator{ \frac 1 \gamma f }{\vecx / \gamma} }
\newcommand{\projBox}[1]{ P_{[a,b]^N}(#1) }
\newcommand{\projBoxSolution}{ \projBox{\vecx} = \text{min}( \text{max} (\vecx, a), b) }
\newcommand{\projLOneTwoBall}[1]{P_{ \{ \LOneTwoNorm{ \cdot } \le \alpha \} }(#1)}
\newcommand{\projLOneTwoBallSolution}{
    (\projLOneTwoBall{\vecx})_{\mathfrak{g}_i} =
        \begin{cases}
            0 & \text{if} \ \LTwoNorm{\vecx_{\mathfrak{g}_i}} = 0, \\
            \bm \beta_i \frac {\vecx_{\mathfrak{g}_i}} {\LTwoNorm{\vecx_{\mathfrak{g}_i}}} & \text{otherwise},
        \end{cases} \\
}
\newcommand{\projLOneTwoBallSolutionWhere}{
    \bm \beta = P_{ \{ \LOneNorm{ \cdot } \le \alpha \} }({[ \LTwoNorm{ \vecx_{\mathfrak{g}_1} }, \ldots, \LTwoNorm{ \vecx_{\mathfrak{g}_N} } ]^T}). \\
}
\newcommand{\projLOneBall}[1]{P_{ \{ \LOneNorm{ \cdot } \le \alpha \} }(#1)}
\newcommand{\projLOneBallSolution}{\projLOneBall{\vecx} = \text{SoftThrethold}(\vecx, \beta)}
\newcommand{\projLOneBallSolutionWhere}{
    \begin{aligned}
        & \bm x_{\text{abs}} = \text{abs}(\vecx), \\
        & \bm y              = \text{sort}_{\text{desc}}(\vecx_{\text{abs}}), \\
        & \beta'             = \text{max} \{ \frac 1 i ((\sum_{j=1}^i \bm y_j) - \alpha) \mid i = 1, \ldots, N \}, \\
        & \beta              = \text{max} \{ \beta', 0 \}. \\
    \end{aligned}
}


% Primal-Dual Splitting
\newcommand{\PDSPrimal}{\minimize { \vecx \in \realNumber^N } \left\{ f(\vecx) + g(\vecx) + h(\bm{L} \vecx) \right\} }
\newcommand{\PDSDual}{\minimize { \vecy \in \realNumber^M } \left\{ (f+g)^*(-\bm{L}^T \vecy) + h^*(\vecy) \right\} }
\newcommand{\PDSSubStep}{
\left \lfloor \ \
    \begin{aligned}
        & \vecx^{(k+1)} = \proximityOperator{\gamma_1 g}{\vecx^{(k)} - \gamma_1( \nabla f(\vecx^{(k)}) + \bm{L}^T \vecy^{(k)} )}, \\
        & \vecy^{(k+1)} = \proximityOperator{\gamma_2 h^*}{\vecy^{(k)} + \gamma_2 \bm{L} (2\vecx^{(k+1)} - \vecx^{(k)}) }, \\
    \end{aligned}
\right.
}

% seismic + related proposed method
\newcommand{\diffOperator}{\mathbf{D}}
\newcommand{\velModel}{\bm{m}}
\newcommand{\seismicData}{\bm{u}}

% FWI objective
\newcommand{\FWIObjectiveDefinition}{ E(\velModel) = \frac {1} {2} \LTwoNorm { \seismicData_{\mathrm{obs}} - \seismicData_{\mathrm{cal}(\velModel)} }^2 }
\newcommand{\FWIGradientDefinition}{ \nabla E(\velModel) = \seismicData_{\mathrm{obs}} - \nabla \seismicData_{\mathrm{cal}(\velModel)} }
\newcommand{\FWIObjectiveWithTVConstraint}{ E(\velModel) \ \ \ \text{s.t.} \ \ \LOneTwoNorm{\diffOperator \velModel} \le \alpha \ , \ \velModel \in [a,b]^N }
\newcommand{\FWIObjectiveWithTVConstraintWithIndicatorFunction}{ E(\velModel) + \indicatorFunction{\LOneTwoNorm{\cdot} \le \alpha}{\diffOperator \velModel} + \indicatorFunction{[a,b]^N}{\velModel} }
\newcommand{\FWIWithPDS}{
\left \lfloor \ \
    \begin{aligned}
        & \widetilde{\velModel}^{(k+1)} = \velModel^{(k)} - \gamma_1( \nabla E(\velModel^{(k)}) + \bm{D}^T \vecy^{(k)} ) \\
        & \velModel^{(k+1)}             = \projBox{\widetilde{\velModel}^{(k+1)}} \\
        & \widetilde{\vecy}^{(k+1)}     = \vecy^{(k)} + \gamma_2 \bm{D} (2\velModel^{(k+1)} - \velModel^{(k)}) \\
        & \vecy^{(k+1)}                 = \widetilde{\vecy}^{(k+1)} - \gamma_2 \projLOneTwoBall{\frac 1 {\gamma_2} {\widetilde{\bm y}^{(k+1)}}}
    \end{aligned}
\right.
}
\newcommand{\FWIWithGradient}{ \velModel^{(k+1)} = \velModel^{(k)} - \gamma( \nabla E(\velModel^{(k)}) ) }





% もし本文が英文ならば,日本語のタイトルと著者は不要です.
% その場合,下記の \OnlyEnglishtrue を有効にしてください.
% 英語のタイトルと著者のみを表示します.
% ただし,\jtitle, \jauthor, \jaddressの行は削字しないでくだ
% さい.

\OnlyEnglishtrue

\jtitle{信号処理シンポジウム}

\etitle{Efficient Full Waveform Inversion Subject To A Total Variation Constraint}

\jauthor{
}

\eauthor{
  Yudai INADA, Shingo TAKEMOTO and Shunsuke ONO
}

\jaddress{
}

\eaddress{
  Institute of Science Tokyo
}

\begin{document}
\maketitle

This paper proposes a computationally efficient algorithm to address the Full-Waveform Inversion (FWI) problem with a Total Variation (TV) constraint, designed to accurately reconstruct subsurface properties from seismic data.
FWI, as an ill-posed inverse problem, requires effective regularizations or constraints to ensure accurate and stable solutions.
Among these, the TV constraint is widely known as a powerful prior for modeling the piecewise smooth structure of subsurface properties.
However, solving the optimization problem is challenging because of the nonlinear observation process combined with the non-smoothness of the TV constraint.
Conventional methods rely on inner loops and/or approximations, which lead to high computational cost and/or inappropriate solutions.
To address these limitations, we develop a novel algorithm based on a primal-dual splitting method, achieving computational efficiency by eliminating inner loops and ensuring high accuracy by avoiding approximations.
We also demonstrate the effectiveness of the proposed method through experiments using the SEG/EAGE Salt and Overthrust Models.
The source code will be available at https://www.mdi.c.titech.ac.jp/publications/fwiwtv.

\section{Introduction}\label{sec:Introduction}
\IEEEPARstart{F}{ull-Waveform} Inversion (FWI)~\cite{FWI0,FWI1} is a technique for reconstructing subsurface properties from seismic data measured at multiple observation points.
The subsurface properties obtained through FWI are essential in geological research and resource exploration, such as locating gas and oil reservoirs, characterizing mineral deposits, and assessing groundwater flow patterns~\cite{FWI1,FWIApplicationGroundwater0,FWIApplicationGroundwater1}.
In addition to the geological field, FWI has also been successfully applied to non-destructive testing, including brain tissue analysis in medical imaging and material detection in industrial inspection ~\cite{FWIApplicationNonDestructiveTesting0,FWIApplicationNonDestructiveTesting1}.

FWI faces significant challenges in directly reconstructing subsurface properties from seismic data because of the nonlinear wave propagation and the inherent complexity of the observation process~\cite{FWI1}.
To address this issue, FWI is typically formulated as an optimization problem that minimizes the squared error between observed and simulated seismic data~\mbox{\cite{FWI0,CustomFWI0,CustomFWI1,CustomFWI2,CustomFWI3,CustomFWI4,CustomFWI5}}, which serves as the standard approach.
Nevertheless, the inherent ill-posedness of FWI necessitates the incorporation of regularization techniques.
Among these, Tikhonov regularization~\cite{tikhonov} and Total Variation (TV)-based methods~\cite{TV,TGV} are widely used to promote piecewise smoothness in the reconstructed subsurface properties, improving stability and accuracy~\cite{FWI-with-tikhonov-regularization,FWI-with-TV-regularization,FWI-with-directional-TV-regularization,FWI-with-high-order-TV-regularization,FWI-with-TGV-regularization}.
While these regularization techniques are effective, they often involve the careful tuning of a balance parameter that determines the trade-off between the FWI objective function and the regularization term.
To overcome this limitation, an alternative approach has been proposed: incorporating the TV prior as a constraint rather than as a regularization term~\cite{FWI-with-TV-constraint,FWI-with-TV-constraint2,FWI-with-TV-constraint3,FWI-with-TV-constraint4}.
This formulation decouples the parameter for the TV constraint from the FWI objective, allowing it to be determined independently based on prior knowledge of subsurface properties~\cite{constraint0,constraint1,constraint2,constraint3,constraint4,constraints-vs-penalties-in-FWI}.
By doing so, this approach not only simplifies parameter selection but also enhances the interpretability of both the mathematical formulation and the reconstructed subsurface models.

Despite its advantages, solving the TV-constrained FWI problem presents considerable difficulties due to the interplay between the nonlinear observation process and the non-smoothness of the TV term.
Conventional methods~\mbox{\cite{FWI-with-TV-constraint,FWI-with-TV-constraint2,FWI-with-TV-constraint3,FWI-with-TV-constraint4}} attempt to address these issues by incorporating inner loops to enforce the constraint at each optimization step or by employing linear or quadratic approximations.
However, these approaches come with notable drawbacks: inner loops substantially increase computational cost, and approximations compromise reconstruction accuracy.
This raises a crucial question: \textit{Is it possible to develop an algorithm that solves the TV-constrained FWI problem efficiently while avoiding inner loops and approximations?}

In this paper, we introduce a novel algorithm for solving the TV-constrained FWI problem using a primal-dual splitting method~\cite{PDS2}.
The proposed algorithm effectively addresses the intertwined issues of the nonlinear observation process and the non-smoothness of the TV constraint, achieving accurate reconstructions without relying on approximations.
Additionally, by eliminating the need for inner loops, our approach is markedly more computationally efficient than existing methods.
We validate the performance of our algorithm through numerical experiments on the SEG/EAGE Salt and Overthrust Models, demonstrating its capability to efficiently enforce the constraints while delivering high-quality reconstructions.


\section{PRELIMINARIES}\label{sec:Preliminaries}
\subsection{Mathematical Tools}\label{subsec:mathematical-tools}

Throughout this paper, we denote vector and matrix by boldface lowercase letter (e.g., $\vecx$) and boldface uppercase letter (e.g., $\bm{X}$), respectively.
The operator $l_{X}$ norm of a vector and matrix is denoted by $\Norm{\cdot}__{X}$.

For $\vecx \in \realNumber^{\intN}$, the mixed $l_{1,2}$ norm is defined as follows:
\begin{equation} \label{eq:L12NormDefinitionEq} \LOneTwoNormDefinition \end{equation}
where $\mathfrak{G}$ is a set of groups, $\vecx_{\mathfrak{g}}$ is the $\mathfrak{g}$-th group of $\vecx$.

For $\vecx \in \realNumber^{\intN}$, the total variation (TV)\cite{TV} is defined as follows:
\begin{equation} \label{eq:TVDefinitionEq} \TotalVariationDefinition \end{equation}
here, $d_{h,i}$ and $d_{v,i}$ are the horizontal and vertical differences of the $i$-th element of $\vecx$, respectively, when vector $\vecx$ is considered as a matrix.

For proper lower-semicontinuous convex function $f \in \realNumber^N \to \realNumber$ and $\vecx \in \realNumber^N$, the convex conjugate function is defined as follows:
\begin{equation} \label{eq:ConjugateFunctionDefinitionEq} \conjugateFunctionDefinition \end{equation}

For a set $C \subset \realNumber^N$ and $\vecx \in \realNumber^N$, the indicator function is defined as follows:
\begin{equation} \label{eq:IndicatorFunctionDefinitionEq} \indicatorFunctionDefinition \end{equation}

For $\gamma > 0$, $f \in \realNumber^N \to \realNumber$ and $\vecx \in \realNumber^N$, the proximity operator is defined as follows:
\begin{equation} \label{eq:ProximityOperatorDefinitionEq} \proximityOperatorDefinition \end{equation}

The proximity operator for the convex conjugate function is expressed as follows\cite[Theorem 3.1 (ii)]{prox-convex-conjugate-function}:
\begin{equation} \label{eq:ProximityOperatorDefinitionWithConvexConjugateFunctionEq} \proximityOperatorDefinitionWithConvexConjugateFunction \end{equation}

The proximity operator for the indicator function is expressed as follows.
\begin{equation} \label{eq:ProximityOperatorDefinitionWithIndicatorFunctionEq}
\proximityOperator{ \gamma \indicatorFunction{C}{\cdot} }{\vecx} = P_C(\vecx) \coloneq \argmin{\vecy \in C} \LTwoNorm{\vecy - \vecx}
\end{equation}

The proximity operator for the box constraint is expressed as follows.
\begin{equation} \label{eq:ProximityOperatorForBoxConstraint} \projBoxSolution \end{equation}

The proximity operator for the $l_1$ norm upper bound constraint is expressed as follows\cite{L1-ball-projection}(faster algorithms are also proposed\cite{fast-L1-ball-projection}):
\begin{equation} \label{eq:ProximityOperatorForL1Ball}  \projLOneBallSolution \end{equation}

The proximity operator for the $l_{1,2}$ norm upper bound constraint is expressed as follows\cite{L12-ball-projection}:
\begin{equation} \label{eq:ProximityOperatorForL12Ball} \projLOneTwoBallSolution \end{equation}

\subsection{Primal-Dual Splitting Algorithm}\label{subsec:primal-dual-splitting-algorithm}
The Primal-Dual Splitting (PDS) Algorithm\cite{PDS0,PDS1,PDS2,PDS3} is applied to the following problem:
\begin{equation} \label{eq:PDSPrimalEq} \PDSPrimal \end{equation}
where $\bm{L} \in \realNumber^{\intM \times \intN}$, $f$ is differentiable convex function and $g,h$ are convex functions whose proximity operator can be computed efficiently(proximable).

The PDS algorithm solve by iteratively updating the following:
\begin{equation} \label{eq:PDSSubStepX} \PDSSubStepX \end{equation}
\begin{equation} \label{eq:PDSSubStepY} \PDSSubStepY \end{equation}
where $\gamma_1, \gamma_2$ are step sizes.
for more details and convergence rates, please refer to\cite{PDS2}.

\subsection{Full Waveform Inversion}\label{subsec:full-waveform-inversion}
In general, an objective function of FWI is defined as follows:
\begin{equation} \label{eq:FWIObjective} \FWIObjectiveDefinition \end{equation}
where $\velModel$ is velocity model, $\seismicData_{\mathrm{obs}}$ is the observed seismic data, and $\seismicData_{\mathrm{cal}(\velModel)}$ is the calculated seismic data with the velocity model.


\section{Proposed Method}\label{sec:ProposedMethod}
As shown in Fig.\ref{fig:salt-model}, the velocity model of the Salt is piecewise smooth.
Therefore, we introduce the TV constraint to achieve more accurate reconstruction.
Also, by introducing the box constraint, we can ensure that the velocity model does not take invalid values, and we show the flexibility of incorporating constraints using PDS.

We minimize the objective function of the TV and box constrained FWI, which is expressed as follows:
\begin{equation} \label{eq:FWIObjectiveWithTVConstraint} \argmin{\velModel \in \realNumber^N} \ \ \FWIObjectiveWithTVConstraint \end{equation}
where $\alpha \in \realNumber$ is the upper bound of the $l_{1,2}$ norm, and $a, b \in \realNumber$ are the lower and upper bounds of the velocity model value, respectively.

The constraints can be incorporated into the objective function as indicator functions:
\begin{equation} \label{eq:FWIObjectiveWithTVConstraintWithIndicatorFunction} \argmin{\velModel \in \realNumber^N} \ \ \FWIObjectiveWithTVConstraintWithIndicatorFunction \end{equation}

The proximity operator of $\iota_{\LOneTwoNorm{\cdot} \le \alpha}$ and $\iota_{[a,b]^N}$ can be computed efficiently.
Therefore, these functions of $E$, $\iota_{[a,b]^N}$ and $\iota_{\LOneTwoNorm{\cdot} \le \alpha}$ correspond to $f$, $g$ and $h$ in \eqref{eq:PDSPrimalEq}, respectively, $\diffOperator$ is corresponds to $\bm{L}$, and the problem~\eqref{eq:FWIObjectiveWithTVConstraintWithIndicatorFunction} can be solved using PDS.
The iterative procedures are as follows:
\begin{equation} \label{eq:FWIWithPDS} \FWIWithPDS \notag \end{equation}


The following are the proximity operators of indicator function of the box constraint and the $l_{1,2}$ norm upper bound constraint.
\begin{equation} \label{eq:ProximityOperatorForBoxConstraint} \projBoxSolution. \end{equation}

%The proximity operator for the $l_{1,2}$ norm upper bound constraint is expressed as follows~\cite{L12-ball-projection}:
\begin{equation} \label{eq:ProximityOperatorForL12Ball} \projLOneTwoBallSolution, \end{equation}
where
\begin{equation} \label{eq:ProximityOperatorForL12BallWhere} \projLOneTwoBallSolutionWhere \notag \end{equation}

The proximity operator for the $l_1$ norm upper bound constraint is expressed as follows~\cite{L1-ball-projection}(faster algorithms are also proposed~\cite{fast-L1-ball-projection}):
\begin{equation} \label{eq:ProximityOperatorForL1Ball}  \projLOneBallSolution, \end{equation}
where
\begin{equation} \label{eq:ProximityOperatorForL1BallWhere} \projLOneBallSolutionWhere \notag \end{equation}




\section{EXPERIMENTS}\label{sec:Experiments}
\begin{figure*}[htbp]
\vspace{-\baselineskip}
    \centering
    \begin{tabular}{m{38mm} m{38mm} m{40mm} m{5mm}}
        \begin{minipage}[b]{40mm}
            \centering
            \includegraphics[width=40mm]{public/true}
            \vspace{-7mm}
            \caption*{\raisebox{2mm}{Background truth}}
        \end{minipage} &
        \begin{minipage}[b]{40mm}
            \centering
            \includegraphics[width=40mm]{public/initial}
            \vspace{-7mm}
            \caption*{\raisebox{2mm}{Initial model}}
        \end{minipage} &
        \begin{minipage}[b]{40mm}
            \centering
            \includegraphics[width=40mm]{public/gradient}
            \vspace{-7mm}
            \caption*{\raisebox{2mm}{Standard FWI}}
        \end{minipage} &
        \multirow[t]{2}{*}{\raisebox{-34mm}{\includegraphics[height=50mm]{public/color-bar}}} \\
        \begin{minipage}[b]{40mm}
            \centering
            \includegraphics[width=40mm]{public/alpha_150}
            \vspace{-8mm}
            \caption*{Proposed Method, $\alpha$=150}
        \end{minipage} &
        \begin{minipage}[b]{40mm}
            \centering
            \includegraphics[width=40mm]{public/alpha_350}
            \vspace{-8mm}
            \caption*{Proposed Method, $\alpha$=350}
        \end{minipage} &
        \begin{minipage}[b]{40mm}
            \centering
            \includegraphics[width=40mm]{public/alpha_550}
            \vspace{-8mm}
            \caption*{Proposed Method, $\alpha$=550}
        \end{minipage} &
    \end{tabular}
    \vspace{-3mm}
    \caption{Velocity models and their corresponding reconstructions.}
    \label{fig:velocity-models}
\vspace{-\baselineskip}
\vspace{2mm}
\end{figure*}

%\begin{figure*}[htbp]
%\vspace{-\baselineskip}
%    \centering
%    \begin{tabular}{m{68mm} m{68mm} m{10mm}}
%        \begin{minipage}[b]{65mm}
%            \centering
%            \includegraphics[width=65mm]{public/true}
%            \vspace{-7mm}
%            \caption*{\raisebox{2mm}{Background truth}}
%        \end{minipage} &
%        \begin{minipage}[b]{65mm}
%            \centering
%            \includegraphics[width=65mm]{public/initial}
%            \vspace{-7mm}
%            \caption*{\raisebox{2mm}{Initial model}}
%        \end{minipage} &
%        \multirow[t]{3}{*}{\raisebox{-87mm}{\includegraphics[height=70mm]{public/color-bar}}} \\
%
%        \begin{minipage}[b]{65mm}
%            \centering
%            \includegraphics[width=65mm]{public/gradient}
%            \vspace{-7mm}
%            \caption*{\raisebox{2mm}{Standard FWI}}
%        \end{minipage} &
%        \begin{minipage}[b]{65mm}
%            \centering
%            \includegraphics[width=65mm]{public/alpha_150}
%            \vspace{-7mm}
%            \caption*{Proposed Method, $\alpha = 150$}
%        \end{minipage} \\
%
%        \begin{minipage}[b]{65mm}
%            \centering
%            \includegraphics[width=65mm]{public/alpha_350}
%            \vspace{-7mm}
%            \caption*{Proposed Method, $\alpha = 350$}
%        \end{minipage} &
%        \begin{minipage}[b]{65mm}
%            \centering
%            \includegraphics[width=65mm]{public/alpha_550}
%            \vspace{-7mm}
%            \caption*{Proposed Method, $\alpha = 550$}
%        \end{minipage} &
%    \end{tabular}
%    \caption{Velocity models and their corresponding reconstructions.}
%    \label{fig:velocity-models}
%\vspace{-\baselineskip}
%\end{figure*}

\begin{figure*}[htbp]
%\vspace{-\baselineskip}
    \centering
    \begin{minipage}{70mm}
        \centering
        \includegraphics[width=\linewidth]{public/alpha-ssim}
        \vspace{-8mm}
        \caption{SSIM against the parameter of alpha.}
        \label{fig:alpha-ssim}
    \end{minipage}
    \hspace{10mm}
    \begin{minipage}{70mm}  % minipageで横並びに
        \centering
        \includegraphics[width=\linewidth]{public/iters-ssim}
        \vspace{-8mm}
        \caption{SSIM against the number of iterations.}
        \label{fig:iters-ssim}
    \end{minipage}
\vspace{-\baselineskip}
\vspace{2mm}
\end{figure*}



\subsection{Experimental Setup}\label{subsec:experimental-setup}

To demonstrate the effectiveness of the TV and box constrained FWI, we conducted experiments where we compared with the standard FWI with gradient method~\eqref{eq:FWIWithGradient}, using the SEG/EAGE Salt and Overthrust Models.
The velocity model consists of 101 $\times$ 51 grid points.
The ground truth velocity model is generated by zooming and cropping Fig.\ref{fig:salt-model}, and the initial velocity model is generated by smoothing the ground truth velocity model with a Gaussian function with a standard deviation of 80.
The number of receivers and source shots are 101 and 20, respectively, and are placed on the surface at equal intervals.
The source waveform is a Ricker wavelet with a peak wavelet frequency of 10 Hz.
The gradient $\nabla E$ is computed numerically using the Devito framework\cite{devito}.
The number of iterations is set to 5000.
In the standard FWI, the step size $\gamma$ is set to $1.0 \times 10^{-4}$.
In the TV and box constrained FWI, the step size $\gamma_1$ and $\gamma_2$ are set to $1.0 \times 10^{-4}$ and $1.0 \times 10^2$, respectively,
    and the lower and upper bounds of the velocity model $a$, $b$ are set to 1.5[km/s] and 4.5[km/s], respectively,
    and experiments are conducted with several $\alpha$ that is the upper bound of the $l_{1,2}$ norm.
%However, it should be noted that parameters such as $\alpha$, $a$, and $b$ were determined by referencing the ground truth data.
%In a practical application, this must be determined independently of this framework.


\subsection{Results and Discussion}\label{subsec:results-and-discussion}

Fig.\ref{fig:velocity-models} shows the ground truth, the initial model, and the reconstructed velocity models using the standard FWI and the TV and box constrained FWI with several $\alpha$.
When $\alpha$ is small, such as 150, the TV constraint is too strong, resulting in an excessively smooth model.
Conversely, when $\alpha$ is large, such as 550, the TV constraint is almost meaningless, and a model similar to the standard FWI is obtained.
When $\alpha$ is appropriate, such as 350, the TV constraint successfully eliminates wave-like artifacts and noise that appear at the source positions, resulting in a more accurate velocity model reconstruction.

%\begin{figure}[htbp]
%%\vspace{-\baselineskip}
%    \begin{center}
%        \includegraphics[width=80mm]{public/alpha-ssim}
%        \vspace{-5mm}
%        \caption{final SSIM against the number of iterations.}
%        \label{fig:alpha-ssim}
%    \end{center}
%    \vspace{-\baselineskip}
%\end{figure}

In Fig.\ref{fig:alpha-ssim}, we plot the Structural Similarity Index Measure (SSIM) at the last iterations against the parameter $\alpha$.
As mentioned earlier, the graph shows that when the value of $\alpha$ is small, the last SSIM decreases, and when the value of $\alpha$ is too large, the results become almost the same as the standard FWI, but not worse.
However, with an appropriately chosen $\alpha$, the graph shows that high SSIM values can be achieved.

%\begin{figure}[htbp]
%%\vspace{-\baselineskip}
%    \begin{center}
%        \includegraphics[width=80mm]{public/iters-ssim}
%        \vspace{-5mm}
%        \caption{SSIM against the number of iterations.}
%        \label{fig:iters-ssim}
%    \end{center}
%    \vspace{-\baselineskip}
%\end{figure}

In Fig.\ref{fig:iters-ssim}, we plot the SSIM against the number of iterations for both methods with $\alpha=350$.
With appropriate parameters, the proposed method consistently achieves higher SSIM values than the standard FWI at every iteration, indicating improved reconstruction accuracy.



\section{CONCLUSION}\label{sec:Conclusion}
In this paper, we developed an efficient algorithm to solve the TV and box constrained FWI problem based on PDS.
Our algorithm does not require approximations when incorporating the constraints, leading to more accurate reconstructions.
Furthermore, the algorithm significantly enhances computational efficiency without inner loops.
Experimental results demonstrate that our method successfully eliminates wave-like artifacts and noise present in the standard FWI method, resulting in a more accurate velocity model and a superior RMSE and SSIM value regardless of the presence of noise in the observed seismic data.


\clearpage

\begin{spacing}{0.85}
    \bibliographystyle{IEEEtran}
    \small
    \setlength{\itemsep}{0pt}
    \bibliography{references}
\end{spacing}

%\begin{thebibliography}{99}
%    \bibitem{full-waveform-inversion} A. Tarantola, ``Inversion of seismic reflection data in the acoustic approximation,'' Geophysics, vol. 49, no. 8, pp. 1259–1266, Aug. 1984.
%    \bibitem{l12-ball-projection} G. Chierchia, N. Pustelnik, J.-C. Pesquet, and B. Pesquet-Popescu, ``Epigraphical projection and proximal tools for solving constrained convex optimization problems,'' Signal, Image and Video Processing, vol. 9, no. 8, pp. 1737–1749, Nov. 2015
%    \bibitem{total-variation} L. I. Rudin, S. Osher, and E. Fatemi, ``Nonlinear total variation based noise removal algorithms,'' Phys. D, vol. 60, no. 1-4, pp. 259–268, Nov. 1992.
%    \bibitem{FWI-total-variation} L. I. Rudin, S. Osher, and E. Fatemi, ``Nonlinear total variation based noise removal algorithms,'' Phys. D, vol. 60, no. 1-4, pp. 259–268, Nov. 1992.
%\end{thebibliography}

\end{document}
